\section{Implementation}
\label{sec:impl}

When \h first runs, it first reads user-specified files that describe the nodes
and the links in the heterogeneous information network.  Optionally, the user
could also specify another set of files that describe the hierarchy of types in
the network so that users could perform drill-down or roll-up OLAP operations
on the network.

Then, \h builds the graph and \hTable, as discussed in
Section~\ref{sec:hier_forest}.  After these two data structures have been
built, the user will be presented with an interactive console. There are 7 key
operations that the user could execute in the console:

\begin{itemize}

    \item \textit{similarity(node1, node2)}: Executes similarity search between
    node1 and node2.

    \item \textit{drill-down(categoryName, categoryValue)}: Executes drill-down
    OLAP operation on the dataset on user specified hierarchy name and value.

    \item \textit{roll-up(categoryName, categoryValue)}: Executes roll-up OLAP
    operation on the dataset on user specified hierarchy name and value.

    \item \textit{add-constraint(type, id)}: Add a constraint on the meta paths
    used in the similarity measures as discussed in
    Section~\ref{sec:user_constraint}.

    \item \textit{delete-constraint(type, id)}: Delete a constraint that the
    user previously added.

    \item \textit{print-meta-paths(node1, node2)}: Display the meta-paths used
    to compute the similarity score between node1 and node2.

    \item \textit{print-network-statistics()}: Display the network statistics
    (i.e., degree distribution, clustering coefficient and average path length)
    of the graph that the user is currently working with.

\end{itemize}
