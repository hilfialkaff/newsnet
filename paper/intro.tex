\section{Introduction}
\label{sec:introduction}

In the current era of big data, an enormous amount of information is being
generated in real-time. It is important that we are able to navigate these
massive datasets to understand what is currently going on. We would like to
specifically focus on news data for a number of reasons. Firstly, there is an
abundance of available news data that we could analyze. Secondly, it is
possible to categorize news data hierarchically. An example of such hierarchy
is the country, followed by state, district, and so on. We should be able to
seamlessly navigate between different level in the hierarchy to retrieve the
relevant information. Last but not least we choose news data because it allows
us to be more aware of the current events that are happening around us. As
such, we would like a platform that allow us to answer questions such as 'Who
are the key politicians for healthcare reforms?', 'How did the opinion of
Barack Obama change over time for education?' and so on.

Similarity search has been extensively studied in the past. At first, it was
applied to only categorical and numerical data types in relational data. Then,
there have been works that tries to leverage link information in networks. Most
of these studies focus on homogeneous or bipartite
networks~\cite{page1999pagerank, jeh2002simrank, xu2007scan}. However,
these similarity measures disregard the subtlety of different types among objects and links
which are present in heterogeneous information network. More recently, a
meta-path based similarity framework, PathSim~\cite{sun2011pathsim}, has been proposed
that takes into account different linkage in the network.

However, there are several shortcomings in the PathSim design.
Firstly, the operations to compute the meta-path are computationally expensive
and consume a lot of memories since it is achieved through matrix multiplications.
Secondly, it does not allow users to perform any OLAP operations in the dataset.
This is important because a user might be interested in the similarity of two
objects under a specified context only. Thirdly, the original PathSim framework
also does not allow user to provide hints in the type of meta-paths that a similarity
search query should return. For instance, a user might be interested in a similarity
search between two objects that does not go into a specific node in the graph.

In this paper, we propose \h, which is built on top of PathSim, that addresses
the aforementioned shortcomings of PathSim. We address the first challenge
by building a hash-table, \mTable, whose key is a pair of node while the value is
the meta-paths between the pair. At first, the hash-table is empty. But as
user queries for similarity measures on two nodes in the graph, the meta-paths
found during the computation will be cached in the hash-table. This
allows for faster indexing if the user performs the same query. For the
second challenge, we address it by (TALK ABOUT FOREST.PY). Finally,
the third challenge is addressed by keeping track of constraints that
the user specifies in a hash-table, \cTable. \cTable will be
checked whenever we found a candidate for meta-path between two objects
to ensure that none of the constraints that the user specified is
violated.

\subsection{Paper Outline}

This paper is structured as follows. We begin with the design of our framework
(Section~\ref{sec:design}) followed by our implementation
(Section~\ref{sec:impl}). We then show our the performance of our framework
based on the DBLP and NYTimes dataset (Section~\ref{sec:eval}). Following that,
we will then survey related works in the area (Section~\ref{sec:rel-work}).
Finally, we discuss some extensions that could further be made to our framework
and conclude (Section~\ref{sec:conc}).
